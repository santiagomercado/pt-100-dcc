\noindent El 20 de mayo de 2019 se puso en vigencia la modificación del Sistema Internacional de Unidades (SI). En el nuevo sistema las unidades de base cambian sus definiciones refiriéndose, en todos los casos, a constantes de referencia. Como Instituto Nacional de Metrología de la República Argentina, el Instituto Nacional de Tecnología Industrial suscribe al nuevo SI y da a conocer a la industria, a las instituciones científicas y a todos los interesados la información de los cambios a través del siguiente enlace https://www.inti.gob.ar/areas/metrologia-y-calidad/si \\\\
\noindent El INTI es el máximo órgano técnico de la República Argentina en el campo de la Metrología. Es función legal del INTI la realización y mantenimiento de los patrones de las unidades de medida, conforme al Sistema Internacional de Unidades (SI), así como su diseminación en los ámbitos de la metrología científica, industrial y legal, constituyendo la cúspide de la pirámide de trazabilidad metrológica en la República Argentina. Los Certificados de Calibración/Medición emitidos por el INTI garantizan la trazabilidad metrológica mediante los patrones nacionales de medida, realizados y mantenidos por el propio INTI
\\\\
\noindent Asimismo, el INTI es firmante del Acuerdo de Reconocimiento Mutuo de Patrones Nacionales de Medida y Certificados de Calibración y Medición (CIPM-MRA), redactado por el Comité Internacional de Pesas y Medidas, por el cual los institutos nacionales de metrología firmantes reconocen entre sí la validez de sus Certificados de Calibración y de Medición para el alcance cubierto por las Capacidades de Medición y Calibración (CMC) incluidas en el Apéndice C de dicho acuerdo, el cual se encuentra disponible en http://kcdb.bipm.org/appendixC/default.asp.
\\\\
\noindent Las CMCs publicadas en la página mencionada son aceptadas por los demás institutos mediante un complejo procedimiento, que incluye una serie de comparaciones internacionales por un lado, por evaluaciones de pares periódicas por otro, y se encuentran soportadas por sistemas de gestión de la calidad basados en la norma ISO/IEC 17025 y en la Norma ISO 17034 cuando corresponde. A la fecha, el INTI posee cerca de 250 capacidades de medición publicadas en el Apéndice C, vinculadas a los servicios de calibración y medición más relevantes. El proceso de declaración y publicación de nuevas CMCs continúa desarrollándose
\\\\
\noindent Por otra parte, el INTI, a través de sus diferentes Unidades Operativas, ubicados en diferentes regiones del país, brinda un Servicio Integrado de Calibración/Medición. En los casos en que diferentes Unidades Operativas ofrecen el mismo servicio, los procedimientos de calibración y medición se encuentran armonizados. De esta manera se acuerdan y establecen internamente metodologías armonizadas para el desarrollo de determinaciones similares y se garantiza la equivalencia y compatibilidad de los resultados.
\\\\
El presente Informe/Certificado está firmado digitalmente mediante Gestión Documental Electrónica (GDE)
cumpliendo con los estándares internacionales de seguridad adoptados por la Infraestructura de FirmaDigital de
la República Argentina (IFDRA).
\vskip 2mm
\hrule
\vskip 1mm
\noindent \textbf{Fin de Certificado}
